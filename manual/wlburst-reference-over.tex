% Womelsdorf Lab burst library function reference - Overview
% Written by Christopher Thomas.

\chapter{Overview}
\label{sect-over}

The wlBurst library functions are divided into several categories:
\begin{itemize}
%
\item \textbf{``Processing''} functions (ch. \ref{sect-proc}) perform
segmentation, feature extraction, and other operations on single data
traces. Events detected are returned as ``event lists''.
%
\item \textbf{``Field Trip''} functions (ch. \ref{sect-ft}) perform event
detection and other operations on Field Trip raw data structures. Events
detected are returned as ``event matrices''. Functions that further process
these event matrices are also provided in this category.
%
\item \textbf{``Synthesis''} functions (ch. \ref{sect-synth}) construct
simulated LFP oscillation events based on supplied parameters. These
functions are also used when reconstructing nominal detected oscillation
waveforms from curve-fit parameters.
%
\item \textbf{``Auxiliary''} functions (ch. \ref{sect-aux}) perform
manipulations that don't fall into the previous categories.
%
\item \textbf{``Plotting''} functions (ch. \ref{sect-plot}) produce rough
plots of various types of data. These are included as part of the sample
code and as an aid to rapid prototyping; the output is generally not
publication-ready.
%
\end{itemize}

Several types of structure and several types of function handle are used by
the library functions. These are described in Chapter \ref{sect-formats}.

%
% This is the end of the file.
